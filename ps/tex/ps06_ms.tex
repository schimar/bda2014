\documentclass[letterpaper, 12pt]{article}
%----------------------------------------
\setlength{\parskip}{1ex plus0.5ex minus0.5ex} 
\setlength{\textheight}{9.0in}
\setlength{\textwidth}{6.5in}
\setlength{\oddsidemargin}{0in}

\addtolength{\topmargin}{-.875in}

\usepackage{amsmath}

\usepackage[pdftex]{graphicx}
\pagestyle{myheadings}
\markright{bda - ps06 - martin schilling}

%---------------------------------------
\begin{document}


\noindent 1) The data for evolutionary change in virus populations was modeled in jags, with: \\
	\indent	$y_{i} \sim Poisson(\lambda_{i})$ \\
linear model:\\
	\indent $ \eta_{i} = \beta1_{i} + \beta2_{i}  + \beta3_{i} * x_{PopSize} + \epsilon_{i}$\\
link function: \\
	\indent $ log(\lambda_{i} = \beta1_{i} + \beta2_{i}  + \beta3_{i} * x_{PopSize} + \epsilon_{i}) $\\

\noindent Priors:\\
	\indent $ \beta1_{j} \sim N(\alpha_{d}, \nu_{d})$\\
	\indent $ \beta2_{j} \sim N(\alpha_{d}, \nu_{d})$\\
	\indent $ \beta3 \sim N(0, 1e^{-6})$\\
	\indent $ \epsilon_{i} \sim N(0, \tau) $\\
		
\noindent with hyperpriors:\\
	\indent $\alpha_{d} \sim N(0, 1e^{-6})$\\
	\indent $ \nu_{d} \sim gamma(0.01, 0.01) $\\
	\indent $ \tau \sim gamma(0.01, 0.01)$\\ 
	

\noindent 2) {\verb JAGS  model \\\begin{verbatim}
model{
	for(i in 1:N){
		## Poisson likelihhod on y
		y[i] ~ dpois(lambda[i])
		## link function and linear model
		log(lambda[i]) <- beta1[strain[i]] + beta2[line[i]] + beta3 * popSize[strain[i]] 
	+ epsi[i]
	}

	for(j in 1:Nstrains){
		## hierarch model for strains 
		beta1[j] ~ dnorm(alpha[1], nu[1])
	}

	for(j in 1:Nlines){
		## hierarch model for lines # with szc; didn't manage to get the szc running
		beta2[j] ~ dnorm(alpha[2], nu[2])
	}
	## beta2[Nlines]<- sum(beta2[1:(Nlines-1)])

	for(k in 1:N){
		## prior on epsilon
		epsi[k] ~ dnorm(0, tau)
	}
	
	for(d in 1:2){
		## Priors on mean and precision for line and strain
		alpha[d] ~ dnorm(0, 1e-6)
		nu[d] ~ dgamma(0.01, 0.01)
	}
	
	## Additional uninformative priors
	beta3 ~ dnorm(0,1e-6)
	tau ~ dgamma(0.01,0.01)
}
\end{verbatim}


\noindent 3) \\
The model was run with 2 chains, a burn-in period of 4000 and 80,000 iterations. The thinning interval was chosen to be 4.\\
The covariate population size was centered. I checked for adequate mixing by graphical means and effective sizes (not shown here).\\

\noindent 4) \\
{\bf (i)} Virion population size affected the molecular substitution rate with a median slope of $1.19e^{-7}$ (with 95\% intervals lying between $-2.18e^{-7}$ and $3.9e^{-7}$)\\
{\bf (ii)} Variation was roughly the same for both mouse lines and virus strains (see Table 1 for median and 95\% intervals).  \\
{\bf (iii)} Mean substitution rates were highest for strain 4 (-2.53, with 95\% ci between -8.68 and 3.54) and lowest for strain 2 (-1.49, with 95\% ci between -7.59 and 4.6). \\

\begin{table}[h]
\caption{Summary statistics for intercepts in generalized linear model. $\beta1_{k}$ = intercept for virus strains; $\beta2_{k}$ = intercept for mouse lines; with k= 5.}
\begin{center}
  \begin{tabular}{ c | l | c | r }
intercept        &         2.5\% &          50\%  &       97.5\% \\ \hline
$\beta1_{1}$&-8.209 &-2.461  &3.916 \\ \hline
$\beta1_{2}$&-7.590 &-1.828  &4.601 \\ \hline
$\beta1_{3}$&-8.293 &-2.614  &3.812 \\ \hline
$\beta1_{4}$&-8.678 &-2.884  &3.539 \\ \hline
$\beta1_{5}$&-8.137 &-2.543  &3.937 \\ \hline
$\beta2_{1}$&-0.8669 & 5.482 &11.220 \\ \hline
$\beta2_{2}$&-1.338 & 5.052  &10.793 \\ \hline
$\beta2_{3}$&-1.354 & 5.010  &10.756 \\ \hline
$\beta2_{4}$&-1.257 & 5.117  &10.838 \\ \hline
$\beta2_{5}$&-0.8825 & 5.489 &11.231 \\ \hline

    \hline 
  \end{tabular}
\end{center}
\end{table}



\end{document}
